\chapter{Συμπεράσματα και Μελλοντική Εργασία}

\section{Συμπεράσματα}

Ας ξεκινήσουμε την εξαγωγή συμπερασμάτων της διπλωματικής εργασίας, ανατρέχοντας στους στόχους που τέθηκαν στο πρώτο κεφάλαιο, αλλά με αντίθετη φορά.
Ιδανικά, θα θέλαμε τα μοντέλα που δημιουργήσαμε να παράγουν κώδικα που <<κάνει>> κάτι χρήσιμο. Αν και αυτό δεν αποκλείεται να συμβεί -- χάρη στη δυνατότητα της προσέγγισης να παράγει κώδικα  επ' άπειρον -- δεν μπορούμε να το εγγυηθούμε, ούτε να το ελέγξουμε.
Εντοπίζουμε λοιπόν την πρώτη αδυναμία της προσέγγισης στην έλλειψη ικανότητας άμεσης οδήγησης των συστημάτων και στη στοχαστική τους φύση.

Έπειτα ενδιαφερόμαστε να εξετάσουμε κατά πόσο ο κώδικας αυτός μπορεί να μεταφραστεί και να μην έχει συντακτικά λάθη.
Αν και μεμονωμένα κομμάτια κώδικα πληρούν αυτές τις προϋποθέσεις, το σύστημα υποπίπτει συχνά σε τέτοια λάθη και κανένα αρχείο δεν παράχθηκε που να είναι συντακτικά και προγραμματιστικά ορθό. 
Ακόμα και όταν τα νευρωνικά δίκτυα ενημερώνονται απλοϊκά (\en{labeled models}) με πρότερη γνώση για τον κώδικα και τη φύση του, δεν εγγυόμαστε πως αποτυπώνουν εύρωστα συμπεράσματα.
Είναι τέτοια η φύση της κατά χαρακτήρα, \en{end-to-end RNN} προσέγγισης που στα πλαίσια του αυτόματου προγραμματισμού αδυνατεί να πετύχει το, γνωστό στη βιβλιογραφία ως, \en{Ground Learning} (Βασική Μάθηση).

Ο απλούστερος στόχος που τέθηκε ήταν το προϊόν των συστημάτων μας να μοιάζει με κώδικα. Αν και στόχος που δεν είναι εύκολο να ποσοτικοποιηθεί, είναι εμφανές τόσο από τα δείγματα όσο και από το σύνολο των παραγώμενων αρχείων πως όλα τα συστήματα που δοκιμάσαμε το καταφέρνουν. 
Σε αντίθεση με το \en{ground learning} η δυνατότητα των συστημάτων για μίμηση είναι εντυπωσιακή.
Ονόματα μεταβλητών που υπακούν στους κανόνες της αγγλικής και τις προγραμματιστικές συμβάσεις, χρήση λογικών δομών και ενδιαφέρουσες αλφαριθμητικές ακολουθίες μαρτυρούν τη αναπαραστατική δύναμη των <<βαθιών>> αναδραστικών νευρωνικών δικτύων.

Σε ότι αφορά την εξέταση των \en{labeled} μοντέλων και της διαίσθησης ότι περισσότερη πληροφορία θα σημαίνει καλύτερα αποτελέσματα κώδικα, η πραγματικότητα είναι διαφορετική.
Η προσέγγιση μας, δεν καταφέρνει να αυξήσει την ποιότητα του παραγώμενου κώδικα, αντίθετα συχνά βοηθάει προς την αντίστροφη κατεύθυνση.
Σημειώνεται πως η ταυτόχρονη δειγματοληψία χαρακτήρα και είδους χαρακτήρα φαίνεται να δυσκολεύει την παραγωγή κώδικα.
Αυτό σημαίνει πως η πρότερη γνώση σύνταξης και σημασιολογίας  θα πρέπει να δοθεί στο σύστημα με πιο αποτελεσματικό τρόπο.

Ο αυτόματος προγραμματισμός είναι ένας εξαιρετικά δύσκολος στόχος. 
Αναμέναμε, λοιπόν, πως οι απλές αυτές προσεγγίσεις με αναδραστικά νευρωνικά δίκτυα δε θα λύσουν το πρόβλημα.
Η προσπάθεια παραγωγής κώδικα όμως, είναι ένας ενδιαφέρον τρόπος να εξετάσουμε την ισχύ τους. 
Χάρη στην πλούσια δυναμική, την δυνατότητα για εκμάθηση σύνθετων αναπαραστάσεων, την αγνωστικότητα ως προς την είσοδο και την έξοδο αλλά και τη δυνατότητα διαχείρισης ακολουθιών είναι ένα σημαντικό εργαλείο στο οπλοστάσιο της Μηχανικής Μάθησης. 


\section{Μελλοντική Εργασία}

Το σύστημα που αναπτύχθηκε στα πλαίσια αυτής της διπλωματικής θα μπορούσε να εξελιχθεί περαιτέρω, τουλάχιστον ως προς τρεις κατευθύνσεις.

Αρχικά έχει ενδιαφέρον η προσπάθεια παραγωγής κώδικα και με διαφορετικές δομές μηχανικής εκμάθησης.
Πιο συγκεκριμένα, πρόσφατα αναπτύχθηκαν τα \en{generative adversarial networks} \cite{Goodfellow2014} (ανταγωνιστικά δίκτυα παραγωγής), δομές στις οποίες το ένα σύστημα νευρωνικών δικτύων παράγει και το δεύτερο κρίνει αυτό που παρήγαγε το πρώτο.
Αυτή η ιδέα πρέπει να δοκιμαστεί στα πλαίσια του αυτόματου προγραμματισμού.

Έπειτα, όπως αναφέραμε, η δομή αναδραστικών δικτύων που χρησιμοποιήθηκε στα πλαίσια της διπλωματικής εργασίας αυτής είναι μία απλοϊκή προσέγγιση.
Το ίδιο ισχύει και για τρόπο με τον οποίο δίνουμε στο σύστημα πρότερη γνώση για τη γλώσσα και τη δομή της.
Δεδομένου του ανεξερεύνητου χώρου έρευνας γύρω από τα νευρωνικά δίκτυα, μπορούμε να δοκιμάσουμε πιο σύνθετες δομές εκμάθησης και να δίνουμε με διαφορετικό τρόπο την πρόσθετη πληροφορία για τον κώδικα.
Μια πρώτη τέτοια ιδέα είναι η χρήση των \en{tree-LSTM} δικτύων \cite{Tai2015} για την διαχείριση των \en{abstract syntax trees} της γλώσσας που χρησιμοποιείται.

Τέλος, η μοντελοποίηση και η πρόβλεψη κατά χαρακτήρα, στα πλαίσια του προγραμματισμού, παρουσιάζουν χρήσιμες ιδιότητες, τις οποίες θα μπορούσαμε να χρησιμοποιήσουμε για να αναπτύξουμε βοηθήματα ανάπτυξης λογισμικού.
Συστήματα στατικής ανάλυσης και διόρθωσης κώδικα αλλά και βοηθήματα συμπλήρωσης κώδικα μπορούν να δημιουργηθούν με βάση τις προσεγγίσεις που περιγράφηκαν παραπάνω.