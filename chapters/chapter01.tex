\chapter{Εισαγωγή}

Η πράξη του προγραμματισμού, δηλαδή η ανάπτυξη μίας διαδικασίας με στόχο την επίτευξη ενός έργου, είναι μια εντυπωσιακή επίδειξη των δυνατοτήτων συλλογιστικής του ανθρώπινου εγκεφάλου.
Η αυτοματοποίηση της συγγραφής κώδικα και προγραμμάτων (Αυτόματος Προγραμματισμός) είναι ένας στόχος με μακρόχρονη ιστορία, τόσο για τους μηχανικούς λογισμικού, όσο και για τον κλάδο της τεχνητής νοημοσύνης.
Ο ακριβής ορισμός του <<Αυτόματου Προγραμματισμού>> παραμένει ένα θέμα στο οποίο υπάρχει ασυμφωνία μεταξύ των ειδικών, γεγονός που ενισχύεται από την συνεχή αλλαγή του όρου χάρη στις εξελίξεις της τεχνολογίας.
Ο \en{David Parnas}, αναζητώντας την ιστορία του όρου, καταλήγει: <<Ο αυτόματος προγραμματισμός ήταν πάντα ένας ευφημισμός για προγραμματισμό σε μια υψηλότερου επιπέδου γλώσσα από αυτή που είναι διαθέσιμη στον προγραμματιστή.>> %TODO Add reference #1

Δεδομένης της εγγενούς δυσκολίας και πολυπλοκότητας του στόχου υπάρχει πληθώρα προκλήσεων αλλά και προσεγγίσεων στη λύση του.
Δύο σημαντικές ομάδες προσεγγίσεων είναι:

\begin{enumerate}
\item Επαγωγικός Προγραμματισμός \en{(Induction Programming)}

Χρησιμοποιώντας τεχνικές τόσο από τον προγραμματισμό όσο και από την τεχνητή νοημοσύνη στοχεύει στη μάθηση προγραμμάτων, τυπικά δηλωτικών και συχνά αναδρομικών.
Για την εκμάθηση χρησιμοποιούνται μη αυστηρές προδιαγραφές, όπως παραδείγματα εισόδου - εξόδου ή περιορισμοί.
\item Παραγωγή Κώδικα Βάσει Οντοτήτων \en{(Ontology Based Code Generation)}

Η συγγραφή του κώδικα γίνεται σύμφωνα με τις οντότητες και τις σχέσεις τους όπως αυτές αποτυπώνονται στο εκάστοτε πρόβλημα.
Εδώ αναφερόμαστε στις απολύτως δομημένες προσεγγίσεις παραγωγής κώδικα όπως αυτές με γνώμονα δομικά διαγράμματα, μοντέλα ή πρότυπα, αλλά και την παραγωγή κώδικα που κάνουν οι μεταφραστές/μεταγλωττιστές γλωσσών προγραμματισμού.
\end{enumerate}

\section{Κίνητρο}
Αφενός, η πρόοδος της τεχνητής νοημοσύνης, και ειδικότερα του κλάδου της υπολογιστικής εκμάθησης \en{(Machine Learning)}, είναι ραγδαία τα τελευταία χρόνια.
Αφετέρου, η εξέλιξη και η ευρεία χρήση του λογισμικού γεννά ανάγκες για αυτοματοποίηση στην παραγωγή του έτσι ώστε να μειωθεί ο χρόνος ανάπτυξης και ο αριθμός λαθών.
Οι σχετικές τεχνολογικές και θεωρητικές ανακαλύψεις ανοίγουν νέα μονοπάτια πειραματισμού, καινούρια εργαλεία αναπτύσσονται και δημιουργούνται κίνητρα επανεξέτασης κάποιων προβλημάτων.

Σύμφωνα με τα παραπάνω και ιδιαίτερα χάρη στις πρόσφατες επιτυχίες γύρω από τα Αναδραστικά Νευρωνικά Δίκτυα, δομές τις οποίες θα εξετάσουμε παρακάτω, είμαστε σε θέση να αντιμετωπίσουμε την παραγωγή κώδικα ως ένα πρόβλημα εκμάθησης ακολουθιών \en{(Sequence Learning)}, προσέγγιση η οποία βρίσκεται ανάμεσα στον επαγωγικό προγραμματισμό και την παραγωγή κώδικα βάσει οντοτήτων.
%TODO Fix one sentence = one paragraph

\section{Περιγραφή του προβλήματος}
Το πρόβλημα μπορεί να αναλυθεί σε δύο διαφορετικά μέρη. 
Από τη μία πλευρά, καλούμαστε να αυτοματοποιήσουμε τη διαδικασία παραγωγής κώδικα. 
Δεδομένων των σύγχρονων μεθόδων και τεχνολογιών, αυτό είναι ζήτημα στο οποίο είναι από εξαιρετικά δύσκολο έως αδύνατο να δοθεί μια γενική λύση, τουλάχιστον για το εγγύς μέλλον.
Αντί για μία γενική λύση, μπορούμε να επικεντρωθούμε στις διεργασίες οι οποίες είναι μεν απλές, αλλά επαναλαμβάνονται συχνά και είναι χρονοβόρες.
Από την άλλη, καλούμαστε να εκμεταλλευτούμε την ραγδαία ανάπτυξη της χρήσης λογισμικού. Έχουμε στη διάθεση μας πληθώρα υλοποιημένων προγραμμάτων, σε πολλές διαφορετικές γλώσσες και μορφές, κάθε δυσκολίας και σκοπού.
Εξίσου πολλά είναι και τα ζευγάρια προβλημάτων - προδιαγραφών.
Ιδανικά, θα θέλαμε να αποφύγουμε να καταβάλουμε κόπο για να δημιουργήσουμε κάτι το οποίο ήδη υπάρχει. 

Η αφθονία της διαθέσιμης πληροφορίας σε μορφή κώδικα -- αλλά και αυτής που εμπεριέχεται σε ολόκληρο το οικοσύστημα του λογισμικού -- δίνει τη δυνατότητα για σχεδιασμό λύσεων οι οποίες έχουν ως επίκεντρο αυτήν ακριβώς τη διαθέσιμη πληροφορία. 
Δεδομένου του εκπαιδευτικού χαρακτήρα της διπλωματικής εργασίας, θα εξετάσουμε το πρόβλημα αυτόματης παραγωγής κώδικα από μία πληροφοριακά οδηγούμενη \en{(data-driven)}σκοπιά, η οποία επιχειρεί να εκμεταλλευτεί τις εξελίξεις στην επιστήμη της πληροφορίας.

\section{Στόχοι της διπλωματικής}

Στόχος της διπλωματικής εργασίας αυτής είναι η δημιουργία ενός τεχνητού νευρωνικού δικτύου με αναδράσεις \en{(artificial recurrent neural network)} το οποίο αφού εκπαιδευτεί στην συγγραφή κώδικα σε μία γλώσσα προγραμματισμού της επιλογής μας -- διαβάζοντας εκατομμύρια γραμμές κώδικα -- θα προσπαθήσει να παράξει κώδικα. Ο κώδικας αυτός γενικά μπορεί να φτάσει σε ένα από τα παρακάτω επίπεδα:

\begin{enumerate}
\item Να <<μοιάζει>> με κώδικα
\item Να μην έχει συντακτικά λάθη
\item Να μπορεί να μεταγλωτιστεί
\item Να <<κάνει κάτι χρήσιμο>>
\end{enumerate}

Σε επίπεδο διπλωματικής εργασίας επιζητούμε κώδικα στα επίπεδα τουλάχιστον 1 ή και 2.


\section{Μεθοδολογία}
Χρησιμοποιούμε μοντέλα βασισμένα σε αναδραστικά νευρωνικά δίκτυα και ένα σύνολο δεδομένων. 
Το τελευταίο αποτελείται από έναν μεγάλο αριθμό προγραμμάτων σε μια γλώσσα της επιλογής μας.
Η μεθοδολογία μπορεί να χωριστεί, αφαιρετικά, σε 3 μέρη:

\begin{enumerate}
\item Προ-επεξεργασία \en{(Pre-processing)}

Δεδομένου ενός μεγάλου όγκου πληροφοριών σε μορφή κώδικα, καλούμαστε να τις επεξεργαστούμε με στόχο την καλύτερη εκμετάλλευση τους από το μοντέλο μας και τελικώς την επίτευξη βέλτιστων αποτελεσμάτων.
Αφαιρούμε την πληροφορία που φαίνεται είτε να δυσκολεύει την εκμάθηση του μοντέλου, είτε είναι αδύνατο να ερμηνευτεί από αυτό.
Σε μία από τις προτεινόμενες προσεγγίσεις προσθέτουμε πληροφορία για τον κώδικα με σκοπό την αποσαφήνιση των δεδομένων.
Η πληροφορία του κώδικα εκφράζεται ως σειρά από στοιχειώδεις χαρακτήρες.

\item Εκπαίδευση \en{(Training)}

Τα προτεινόμενα μοντέλα, τα οποία είναι σύνθετες δομές αναδραστικών νευρωνικών δικτύων, εκπαιδεύονται βάσει της παραπάνω επεξεργασμένης πληροφορίας.
Μετά από το "διάβασμα" μιας σειράς χαρακτήρων καλούνται να προβλέψουν τον επόμενο χαρακτήρα. Οι επιδόσεις εκφράζονται μέσω μιας μετρικής λάθους, την οποία η εκπαιδευτική διαδικασία προσπαθεί να ελαχιστοποιήσει χρησιμοποιώντας γενικευμένες μεθόδους βελτιστοποίησης.  


\item Παραγωγή κώδικα \en{(Code Generation)}

Τα εκπαιδευμένα, πλέον, μοντέλα μπορούν να χρησιμοποιηθούν για την παραγωγή κώδικα.
Αρχικοποιούνται με κώδικα της επιλογής μας, ο οποίος επεξεργάζεται όπως και τα δεδομένα στα οποία εκπαιδεύεται.
Το μοντέλο παράγει ένα χαρακτήρα σε κάθε πρόβλεψη και χρησιμοποιεί την πρόβλεψη του ως αληθή για να παράξει τον επόμενο χαρακτήρα.
Με αυτό τον τρόπο μπορεί να συγγράφει απεριόριστη ποσότητα κώδικα.
\end{enumerate}

\section{Διάρθρωση}
Η εργασία αυτή είναι οργανωμένη σε πέντε κεφάλαια: Στο Κεφάλαιο 2
δίνεται το θεωρητικό υπόβαθρο των βασικών τεχνολογιών που
σχετίζονται με τη διπλωματική αυτή. Αρχικά περιγράφονται ..., στη συνέχεια το ... και τέλος ... . 
Στο κεφάλαιο 3 παρουσιά
Στο Κεφάλαιο 4 αρχικά παρουσιάζεται ανάλυση και η σχεδίαση του συστήματος ... .. Τέλος στο Κεφάλαιο 5 δίνονται τα συμπεράσματα, η συνεισφορά αυτής της
διπλωματικής εργασίας, καθώς και μελλοντικές επεκτάσεις.