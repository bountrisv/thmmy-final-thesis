\chapter{Εισαγωγή}

Η πράξη του προγραμματισμού, δηλαδή η ανάπτυξη μίας διαδικασίας με στόχο την επίτευξη ενός έργου, είναι μια εντυπωσιακή επίδειξη των δυνατοτήτων συλλογιστικής του ανθρώπινου εγκεφάλου.
Η αυτοματοποίηση της συγγραφής κώδικα και προγραμμάτων (Αυτόματος Προγραμματισμός) είναι ένας στόχος με μακρόχρονη ιστορία, τόσο για τους μηχανικούς λογισμικού, όσο και για τον κλάδο της τεχνητής νοημοσύνης.
Ο ακριβής ορισμός του <<Αυτόματου Προγραμματισμού>> παραμένει ένα θέμα στο οποίο υπάρχει ασυμφωνία μεταξύ των ειδικών, γεγονός που ενισχύεται από την συνεχή αλλαγή του όρου χάρη στις εξελίξεις της τεχνολογίας.
Ο \en{David Parnas}, αναζητώντας την ιστορία του όρου, καταλήγει: <<Ο αυτόματος προγραμματισμός ήταν πάντα ένας ευφημισμός για προγραμματισμό σε μια υψηλότερου επιπέδου γλώσσα από αυτή που είναι διαθέσιμη στον προγραμματιστή.>> \cite{Parnas1985}

Δεδομένης της εγγενούς δυσκολίας και πολυπλοκότητας του στόχου υπάρχει πληθώρα προκλήσεων αλλά και προσεγγίσεων στη λύση του.
Δύο σημαντικές ομάδες προσεγγίσεων είναι \cite{Biermann1985}, \cite{Schmidt2006}:

\begin{enumerate}
\item Επαγωγικός Προγραμματισμός \en{(Induction Programming)} 

Χρησιμοποιώντας τεχνικές τόσο από τον προγραμματισμό όσο και από την τεχνητή νοημοσύνη στοχεύει στη μάθηση προγραμμάτων, τυπικά δηλωτικών και συχνά αναδρομικών.
Για την εκμάθηση χρησιμοποιούνται μη αυστηρές προδιαγραφές, όπως παραδείγματα εισόδου - εξόδου ή περιορισμοί.
\item Παραγωγή Κώδικα Βάσει Μοντέλων \en{(Model-Driven Code Generation)}

Στην προσπάθεια των ερευνητών λογισμικού να απλοποιήσουν την <<διαδρομή>> ανάμεσα στη σχεδίαση και την υλοποίηση χρησιμοποίουνται αφαιρέσεις.
Ένα σύνολο τεχνικών με ευρεία και αυξανόμενη χρήση, το οποίο βασίζεται σε τέτοιες αφαιρέσεις, είναι το \en{Model Driven Engineering}.
Γλώσσες μοντελοποίησης που αφορούν συγκεκριμένους τομείς \en{Domain-specific modeling languages)} χρησιμοποιούνται σε συνδυασμό με συστήματα μετατροπών και παραγωγής \en{(transormation engines and generators)} για να φτιάξουν αντικείμενα όπως κώδικα, προσομοιώσεις εισόδου ή και άλλα μοντέλα. 
\end{enumerate}

Με ένα λειτουργίκο σύστημα αυτόματης παραγωγής κώδικα ο χρόνος ανάπτυξης και ο αριθμός των λαθών μειώνεται δραματικά. 
Αντίστροφα, εκτινάσσεται η παραγωγικότητα των χρηστών και απλουστεύεται η αντιμετώπιση σύνθετων προβλημάτων.

\section{Κίνητρο}
Αφενός, η πρόοδος της τεχνητής νοημοσύνης, και ειδικότερα του κλάδου της υπολογιστικής εκμάθησης \en{(Machine Learning)}, είναι ραγδαία τα τελευταία χρόνια.
Αφετέρου, η εξέλιξη και η ευρεία χρήση του λογισμικού δημιουργεί ανάγκες για αυτοματοποίηση στην παραγωγή του αλλά και γιγαντιαία ποσότητα υλικού το οποίο μπορούμε να χρησιμοποιήσουμε.
Έχουμε στη διάθεση μας πληθώρα υλοποιημένων προγραμμάτων, σε πολλές διαφορετικές γλώσσες και μορφές, κάθε δυσκολίας και σκοπού.
Οι σχετικές τεχνολογικές και θεωρητικές ανακαλύψεις ανοίγουν νέα μονοπάτια πειραματισμού, καινούρια εργαλεία αναπτύσσονται και δημιουργούνται κίνητρα επανεξέτασης κάποιων προβλημάτων.

Σύμφωνα με τα παραπάνω, και ιδιαίτερα χάρη στις πρόσφατες προόδους της υπολογιστικής εκμάθησης γύρω από την ταξινόμηση και παραγωγή κειμένου \cite{Graves2013}, \cite{Liu2016}, καλούμαστε να εξετάσουμε πως και σε τι βαθμό μπορούμε να τις εκμεταλλευτούμε ως μηχανικοί λογισμικού. Τι εφαρμογές μπορούν να προκύψουν για την πρόβλεψη και τη διόρθωση κώδικα$;$ Μέχρι ποιο σημείο μπορούμε να αυτοματοποιήσουμε την παραγωγή του$;$

\section{Περιγραφή του προβλήματος}
Το πρόβλημα που τίθεται προς λύση είναι η αυτοματοποίηση της παραγωγής κώδικα. 
Δεδομένων των σύγχρονων μεθόδων και τεχνολογιών, αυτό είναι ζήτημα στο οποίο είναι από εξαιρετικά δύσκολο έως αδύνατο να δωθεί μια γενική λύση, τουλάχιστον για το εγγύς μέλλον.
Αντί για μία γενική λύση, μπορούμε να επικεντρωθούμε στις διεργασίες οι οποίες είναι μεν απλές, αλλά επαναλαμβάνονται συχνά και είναι χρονοβόρες.
Ιδανικά, θα θέλαμε να αποφύγουμε να καταβάλουμε κόπο για να δημιουργήσουμε κάτι το οποίο ήδη υπάρχει. 

\section{Στόχοι της διπλωματικής}

Στόχος της διπλωματικής εργασίας αυτής είναι η δημιουργία ενός τεχνητού νευρωνικού δικτύου με αναδράσεις \en{(artificial recurrent neural network)} το οποίο αφού εκπαιδευτεί στην συγγραφή κώδικα σε μία γλώσσα προγραμματισμού της επιλογής μας -- διαβάζοντας εκατομμύρια γραμμές κώδικα -- θα προσπαθήσει να παράξει κώδικα. Δεδομένου του εκπαιδευτικού χαρακτήρα της διπλωματικής εργασίας, θα εξετάσουμε το πρόβλημα αυτόματης παραγωγής κώδικα από μία πληροφοριακά οδηγούμενη \en{(data-driven)} σκοπιά, η οποία επιχειρεί να εκμεταλλευτεί τις εξελίξεις στην επιστήμη της πληροφορίας.

Ο κώδικας αυτός γενικά μπορεί να φτάσει σε ένα από τα παρακάτω επίπεδα:

\begin{enumerate}
\item Να <<μοιάζει>> με κώδικα
\item Να μην έχει συντακτικά λάθη
\item Να μπορεί να μεταγλωτιστεί
\item Να <<κάνει κάτι χρήσιμο>>
\end{enumerate}

Σε επίπεδο διπλωματικής εργασίας επιζητούμε κώδικα στα επίπεδα τουλάχιστον 1 ή και 2.

\section{Μεθοδολογία}
Θα αντιμετωπίσουμε την παραγωγή κώδικα ως ένα πρόβλημα εκμάθησης ακολουθιών \en{(Sequence Learning)}, προσέγγιση η οποία βρίσκεται ανάμεσα στον επαγωγικό προγραμματισμό και την παραγωγή κώδικα βάσει μοντέλων.
Χρησιμοποιούμε μοντέλα βασισμένα σε αναδραστικά νευρωνικά δίκτυα και ένα σύνολο δεδομένων. 
Το τελευταίο αποτελείται από έναν μεγάλο αριθμό προγραμμάτων σε μια γλώσσα της επιλογής μας. Σε αυτή την περίπτωση θα χρησιμοποιήσουμε τη γλώσσα javascript, αλλά το μοντέλο μας είναι αγνωστικό στο ποια γλώσσα μαθαίνει.
Η μεθοδολογία μπορεί να χωριστεί, αφαιρετικά, σε 3 μέρη:

\begin{enumerate}
\item Προ-επεξεργασία \en{(Pre-processing)}

Δεδομένου ενός μεγάλου όγκου πληροφοριών σε μορφή κώδικα, καλούμαστε να τις επεξεργαστούμε με στόχο την καλύτερη εκμετάλλευση τους από το μοντέλο μας και τελικώς την επίτευξη βέλτιστων αποτελεσμάτων.
Αφαιρούμε την πληροφορία που φαίνεται είτε να δυσκολεύει την εκμάθηση του μοντέλου, είτε είναι αδύνατο να ερμηνευτεί από αυτό.
Σε μία από τις προτεινόμενες προσεγγίσεις προσθέτουμε πληροφορία για τον κώδικα με σκοπό την αποσαφήνιση των δεδομένων.
Η πληροφορία του κώδικα εκφράζεται ως σειρά από στοιχειώδεις χαρακτήρες.

\item Εκπαίδευση \en{(Training)}

Τα προτεινόμενα μοντέλα, τα οποία είναι σύνθετες δομές αναδραστικών νευρωνικών δικτύων, εκπαιδεύονται βάσει της παραπάνω επεξεργασμένης πληροφορίας.
Μετά από το "διάβασμα" μιας σειράς χαρακτήρων καλούνται να προβλέψουν τον επόμενο χαρακτήρα. Οι επιδόσεις εκφράζονται μέσω μιας μετρικής λάθους, την οποία η εκπαιδευτική διαδικασία προσπαθεί να ελαχιστοποιήσει χρησιμοποιώντας γενικευμένες μεθόδους βελτιστοποίησης.  


\item Παραγωγή Κώδικα \en{(Source Code Generation)}

Τα εκπαιδευμένα, πλέον, μοντέλα μπορούν να χρησιμοποιηθούν για την παραγωγή κώδικα.
Αρχικοποιούνται με κώδικα της επιλογής μας, ο οποίος επεξεργάζεται όπως και τα δεδομένα στα οποία εκπαιδεύεται.
Το μοντέλο παράγει ένα χαρακτήρα σε κάθε πρόβλεψη και χρησιμοποιεί την πρόβλεψη του ως αληθή για να παράξει τον επόμενο χαρακτήρα.
Με αυτό τον τρόπο μπορεί να συγγράφει απεριόριστη ποσότητα κώδικα.
\end{enumerate}

\section{Διάρθρωση}
Η εργασία αυτή είναι οργανωμένη σε πέντε κεφάλαια: Στο Κεφάλαιο 2
δίνεται το θεωρητικό υπόβαθρο των βασικών τεχνολογιών που
σχετίζονται με τη διπλωματική αυτή. Αρχικά περιγράφονται ..., στη συνέχεια το ... και τέλος ... . 
Στο κεφάλαιο 3 παρουσιά
Στο Κεφάλαιο 4 αρχικά παρουσιάζεται ανάλυση και η σχεδίαση του συστήματος ... .. Τέλος στο Κεφάλαιο 5 δίνονται τα συμπεράσματα, η συνεισφορά αυτής της
διπλωματικής εργασίας, καθώς και μελλοντικές επεκτάσεις.