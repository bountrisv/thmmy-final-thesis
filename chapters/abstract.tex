\begin{acknowledgements}
Θα ήθελα αρχικά να ευχαριστήσω τον καθηγητή κ. Ανδρέα Συμεωνίδη για την εμπιστοσύνη και την καθοδήγηση 
και τον κ. Κυριάκο Χατζηδημητρίου για την καθοδήγηση και τη συνεργασία στα πλαίσια αυτής της διπλωματικής εργασίας. Επίσης θέλω να ευχαριστήσω την οικογένειά μου για την αδιάληπτη στήριξη όλα αυτά τα χρόνια.
\end{acknowledgements}

\begin{abstract}
Η περίληψη θα συμπληρωθεί αργότερα. Αυτή είναι μια περίληψη άλλης εργασίας:

Ένα σύστημα ομότιμων κόμβων αποτελείται από ένα σύνολο αυτόνομων
υπολογιστικών κόμβων στο Διαδίκτυο, οι οποίοι συνεργάζονται με
σκοπό την ανταλλαγή δεδομένων. Στα συστήματα ομότιμων κόμβων που
χρησιμοποιούνται ευρέως σήμερα, η αναζήτηση πληροφορίας γίνεται με
χρήση λέξεων κλειδιών. Η ανάγκη για πιο εκφραστικές λειτουργίες,
σε συνδυασμό με την ανάπτυξη του Σημασιολογικού Ιστού, οδήγησε στα
συστήματα ομότιμων κόμβων βασισμένα σε σχήματα. Στα συστήματα αυτά
κάθε κόμβος χρησιμοποιεί ένα σχήμα με βάση το οποίο οργανώνει τα
τοπικά διαθέσιμα δεδομένα. Για να είναι δυνατή η αναζήτηση
δεδομένων στα συστήματα αυτά υπάρχουν δύο τρόποι. Ο πρώτος είναι
όλοι οι κόμβοι να χρησιμοποιούν το ίδιο σχήμα κάτι το οποίο δεν
είναι ευέλικτο. Ο δεύτερος τρόπος δίνει την αυτονομία σε κάθε
κόμβο να επιλέγει όποιο σχήμα θέλει και απαιτεί την ύπαρξη κανόνων
αντιστοίχισης μεταξύ των σχημάτων για να μπορούν να αποτιμώνται οι
ερωτήσεις. Αυτός ο τρόπος προσφέρει ευελιξία όμως δεν υποστηρίζει
την αυτόματη δημιουργία και τη δυναμική ανανέωση των κανόνων, που
είναι απαραίτητες για ένα σύστημα ομότιμων κόμβων.

Στόχος της διπλωματικής εργασίας είναι η ανάπτυξη ενός συστήματος
ομότιμων κόμβων βασισμένο σε σχήματα το οποίο (α) θα επιτρέπει μια
σχετική ευελιξία στην χρήση των σχημάτων και (β) θα δίνει την
δυνατότητα μετασχηματισμού ερωτήσεων χωρίς την ανάγκη διατύπωσης
κανόνων αντιστοίχισης μεταξύ σχημάτων, xρησιμοποιώντας κόμβους με
σχήματα \tl{RDF} που αποτελούν υποσύνολα-όψεις ενός βασικού
σχήματος (καθολικό σχήμα).


   \begin{keywords}
   Σύστημα ομότιμων κόμβων, Σύστημα ομότιμων κόμβων βασισμένο σε
   σχήματα, Σημασιολογικός Ιστός, \tl{RDF/S}, \tl{RQL}, \tl{Jxta}
   \end{keywords}
\end{abstract}



\begin{abstracteng}
\tl{This is a placeholder for the abstract of my thesis. The actual abstract is going to be written after I finish all chapters. The Compact LInear Collider (CLIC) will use a novel acceleration scheme in which energy extracted from a very intense beam of relatively low-energy electrons (the Drive Beam) is used to accelerate a lower intensity Main Beam to very high energy. The high intensity of the Drive Beam, with pulses of more than $10^{15}$ electrons, poses a challenge for conventional profile measurements such as wire scanners. Thus, new non-invasive profile measurements are being investigated.}

\tl{One candidate is the Electron Beam Scanner. A probe beam of low-energy electrons crosses the accelerator beam perpendicularly. The probe beam is deflected by the space-charge fields of the accelerator beam. By scanning the probe beam and measuring its deflection with respect to its initial position, the transverse profile of the accelerator beam can be reconstructed.}

\tl{Analytical expressions for the deflection exist in the case of long bunches, where the charge distribution can be considered constant during the measurement. In this paper we consider the performance of an electron beam scanner in an accelerator where the bunch length is much smaller than the probe-beam scanning time. In particular, the case in which the bunch length is shorter than the time taken for a particle of the probe beam to cross the main beam is difficult to model analytically. We have developed a simulation framework allowing this situation to be modelled.}

   \begin{keywordseng}
    \tl{Fill in }
    %TODO fill in keywords
   \end{keywordseng}

\end{abstracteng}
