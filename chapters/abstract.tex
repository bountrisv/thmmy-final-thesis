\begin{acknowledgements}
Θα ήθελα αρχικά να ευχαριστήσω τον επίκουρο καθηγητή κ. Ανδρέα Συμεωνίδη για την εμπιστοσύνη και την καθοδήγηση.
Έπειτα θα ήθελα να ευχαριστήσω τον μεταδιδακτορικό ερευνητή κ. Κυριάκο Χατζηδημητρίου για την καθοδήγηση και τη συνεργασία.
Τέλος θέλω να ευχαριστήσω την οικογένειά μου για την αδιάλειπτη στήριξη.
\end{acknowledgements}

\begin{abstract}

Η εξέλιξη των κλάδων της Μηχανικής Εκμάθησης και της Επιστήμης της Πληροφορίας είναι ραγδαία την τελευταία δεκαετία.
Ως μηχανικοί λογισμικού, αναζητούμε τρόπους να εκμεταλλευτούμε την εξέλιξη αυτή.

Στην παρούσα διπλωματική εργασία εξετάζουμε την ικανότητα των αναδραστικών νευρωνικών δικτύων στην παραγωγή κώδικα, ως δίκτυα ικανά να διαχειριστούν ακολουθίες.
Προτείνουμε δύο προσεγγίσεις, που βασίζονται στην κατά χαρακτήρα ανάλυση αποθετηρίων κώδικα, με γλώσσα επιλογής την \en{javascript}.
Μετά την κατάλληλη προ-επεξεργασία του κώδικα και την εκπαίδευση των δικτύων, τα μοντέλα μπορούν να παράγουν κώδικα μέσω μιας στοχαστικής διαδικασίας.
Με σκοπό την εξέταση των επιδόσεων των προσεγγίσεων και των επιμέρους μοντέλων, εκτελούμε στατική ανάλυση κώδικα στα προϊόντα τους.

Η ανάλυση δείχνει την μεγάλη αναπαραστατική δύναμη των αναδραστικών νευρωνικών δικτύων αλλά και την αδυναμία των προσεγγίσεων μας να αντιμετωπίσουν το πρόβλημα του αυτόματου προγραμματισμού. Με βάση αυτά τα ευρήματα προτείνουμε περαιτέρω ερευνητικές κατευθύνσεις και τρόπους εκμετάλλευσης των μοντέλων που σχεδιάστηκαν.  

   \begin{keywords}
	Αναδραστικά Νευρωνικά Δίκτυα, Παραγωγή Κώδικα, Χαρακτήρας, Ακολουθία, Εκμάθηση
   \end{keywords}
\end{abstract}



\begin{abstracteng}
\selectlanguage{english}
The evolution of Machine Learning and Data Science disciplines has been rapid in the last decade.
As software engineers, we are looking for ways to take advantage of this evolution.

In this diploma thesis we examine the ability of recurrent neural networks to generate code, because of their natural ability to handle sequences.
We propose two approaches, based on per-character analysis of software repositories. The language we choose to work with is JavaScript.
Following appropriate code pre-processing and network training, models can generate code through a stochastic process.
In order to examine the performance of the approaches and the individual models, we perform static code analysis on their products.

The analysis shows the great representational power of the recurrent neural networks and the inability of our approaches to address the problem of automatic programming. Based on these findings, we propose further research directions and ways of exploiting the models that were designed.

   \begin{keywordseng}
    Recurrent Neural Networks, Source Code Generation, Character, Sequence, Learning
   \end{keywordseng}

\selectlanguage{greek}
\end{abstracteng}